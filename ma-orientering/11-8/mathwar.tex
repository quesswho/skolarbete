\chapter{The mathematics of war!}
\section{notes}
\enum{
\item By Sean Gourley

}
Afghanistan and iraq war, sierra lon...

Sean Gourley new zealander physisist.
Was watching oxford news and started gathering information.\\ 
Stringing together a team of scientists, economists and mathematicians. \\
Gathering raw data from a bunch of organizations.\\
Gathering for example the number of people killed by the number of attacks.\\
If you take all the attacks, you get a really good mathematical model which describes everything. \\
They looked a few more conflicts and even with different economic problems different cultures and the same mathematical pattern appeard. The diststribution was always the same.
They could create an equation that would predict the number of people $P(x)=Cx^{-\alpha}$ Probabilty P, x number of people killed, and two constants C, and $\alpha$ is organizational structure. Coalescence + Fragmentation. Can we simulate it and it turns out we can, using root dynamics to describe this.
Why do these rather complex have all such a similar pattern.
How do we end the war?
We can change the structure which in turn changes the $\alpha$ value. We should look at the structure of the insurgency if we want to win the war.         

Where $P$ is the probability of an attack with $x$ deaths.
\section{Sammanfattning}
In Sean Gourley's TED conference titled "The Mathematics of War" (2009) Gourley talks about his research in creating a mathematical model which can describe the probabilistic nature of attacks.\\\\
Stringing together a team of scientists, economists and mathematicians suitable for  