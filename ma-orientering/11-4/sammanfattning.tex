\chapter{Can maths combat terrorism?}
Terrorism is a human activity that may seem random and complex, studying the bigger picture reveals hidden patterns. 
Mathematicians have been able to use the power law distribution curve to model the frequency of large and small attacks.\\\\
Political sciences had been studying terrorism for decades. Political sciences have been viewing terrorism as a phenomena of decisions by studying the motives and culture of the terrorists. In contrast the mathematicians are viewing it as a natural phenomena by looking at the bigger picture and studying the trends of terrorism. \\\\
The mathematical model is not able to predict exactly when a terrorist attack will occur.
However, by finding trends, the model can predict the frequency and make probabilistic statements about the size of an attack within a rough time frame.
\\\\
A mathematical model called predictive policing helps police forecast where crime is most likely to occur. The software produces a heat map which highlights areas of interest. After the adaptation of this software crime has fallen by 30\%. By using 7 years of data the algorithm can produce useful results.\\\\
Very unlikely that there will ever be a model that can predict terrorist attacks exact time and scale, because humans are unpredictable, and to get a sense of how they will act in the future is incredibly hard. Humans are complex and irrational and are very hard to simulate due to their randomness. \\\\
\section{Noteringar}
Referara sådan att det inte ska vara själv utvärderande.