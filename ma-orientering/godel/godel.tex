\chapter{ngt}
\section{Matematikens grunder}
\enum{
\item Gränsvärde
\item Kontinuitet
\item Vadå är en funktion?
\item  Hur kan såda defineras?
\item Hur kan vi definera reela talen?
}
\textbf{Diagonalargumentet}\\
George Cantor formulerade det cirka 1870-talet
Flera reella tal än heltal.\\
Bertrand Russel Paradox:\\
 Låt $G(x)$ betyda $\neg x(x)$
$$\forall x G(x) \iff \neg x(x)$$ 
Sätt x=G:
$$G(G) \iff \neg G(G)$$
Motsägelse.

Mängdlära:\\
$\{x:x\not \in x\}=R$
Gäller att $R\in R$? Det blir en motsägelse.\\\\
Principia Mathematica 1900-1920 (Russel \& Whitehead)\\\\
Kurt Gödel (1930): Ofullständighetssatsen: Om Pricipia Mathematica är motsägelsefritt, så är det ofullständigt.\\
Kurt Gödel är den ända personen som har läst hela Principia Mathematica.\\\\
Alan Turing 1912-1954
Turingmaskin - Matematisk modell av dator.