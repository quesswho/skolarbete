\subsection*{Avslutning}
Trots talteori är ren matematisk bransch, en bransch som utvecklas utan den primära tanken att tillämpas till samhället, har den även tillämpningar. Till exempel RSA-kryptosystemet som kan användas för att kryptera och avkryptera meddelanden, som troligtvis kommer att fortsätta att användas flera år i framtiden. Talteori har även en roll inom teoretisk datorvetenskap, för att lösa heltals ekvationer, till exempel för att lösa linjära modulära ekvationen kan man använda utökad Euklides algoritm. Andra grenar inom matematiken har även en roll inom talteori, till exempel analys, det första beviset av primtalssatsen användes komplex analys som ett hjälpmedel, och i exemplet med Riemann zeta-funktionen användes gammafunktionen för att representera den talteoretiska zeta-funktionen. Ett annat exempel algebraisk talteori, där man använder algebraiska strukturer för att generalisera och formulera talteori på nya sätt, med algebraiska terminologi. Här pågår forskning inom till exempel Langlands teori där galoisteori har en väsentlig roll.
% Hösten 2022