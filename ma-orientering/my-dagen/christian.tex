\chapter{Christian Johansson - Att Forska i matematik}
\textit{Att bli en matematiker}\\

\noindent Christian Johansson är doktorand i Imperial College London. Lektor i GU/CTH. Forskning inom algebraisk talteori (langlandsteorin)
Ex
$$y^2=x^3-16x+16 \mod{p}$$
Lösning ges av $p-a_p$, där $a_p$ definieras av
$f(z)$ något...
f är en modulär form \textemdash en speciell typ av komplex funktion.
Bara ett exempel av många. \\Langlandsprogrammet söker ett systematiskt svar på varför det här sker.
Centralt är Galoisteori, matematiken som utvecklades för att visa att femtegradsekvationer inte kan lösas.\\
\textbf{Löser existerande problem}
\enum{
\item Oftast nyare frågeställningar \textemdash Matematiken utvecklas ständigt
\item  Ibland äldre \textemdash Fermat stora sats, Reimannhypotesen, P=NP, Navier \textemdash Stokes ekvationer, ...
\item Gamla problem fungerar som motivation och som måttstockar för framsteg
}

\textbf{Hittar nya kopplingar}
\enum{
\item  Smått \textemdash problemlösning kräver ofta att man relaterar ett olöst problem till ett löst problem
\item Stort \textemdash kopplingar mellan hela fält

}

\textbf{Skapar ny matematik:}
\enum{
\item Det är fritt för alla att skapa ny matematik så länge man följer logikens lagar.
\item Fascinerande i sig självt, eller säger något nytt om den matematik som redan finns.
}

\textbf{Vad bestämmer vilken forskning som sker?}
\enum{
\item Trender
\item Tradition
\item Tillämpningar
\item Nyfikenhet
}

$$\text{Maetmatisk Teori}\iff \text{Gamla problem} \iff \text{Nya problem}$$

\enum{
\item Statistical inference on interacting particle systems
\item  Rational lines on cubic hyper surfaces
}
\textbf{Hur ser en dag ut?}
Man sitter och klurar. Länge. I grupp eller ensam
Läser och diskuterar med kollegor. Prova sig fram. Penna och papper. Kanske dator.

Också:
\enum{
\item Undervisar
\item Skriver forskningsartiklar
\item Organiserar eminarier och konferenser
\item Etc
}

\section{Karriärsväg}
\textbf{Doktorerar}
\enum{
\item Nå forskningsfronten
\item 3-5 år, mestadels forskning, samt undervisning och studier.
\item Lärlingsutbildning \textemdash Lära sig forska under handledning
}

\textbf{Postdoc}
\enum{
\item Etablera sig som forskare
\item Tidsbegränsade tjänser, normalt 1-3 år per tjänst
\item Ofta flera på olika ställn, forskning och undervisining
}

\textbf{Tillsvidareanställning}
\enum{
\item Mängden forskning variera från fall till fall
}


\textbf{Varför är jag matematiker?}
\enum{
\item Ett stort matematikintresse
\item Ett jobb med stor frihet
\item En internationell arbetsmiljö
}

Lästips:
\enum{
\item Fermats gåta, S. Singh
\item Birth of a theorem, C. Villani
\item The Millenium Problems, K. Devlin
}

Man ska inte börja på ett problem om man inte har någon ide på hur man skulle kunna påbörja

\chapter{Sammanfattning}
Christian Johansson is a professional mathematician and lecturer at the University of Gothenburg and Chalmers Technical University. Johansson received his PhD at the Imperial College in London. He is currently researching algebraic number theory (Langlandstheory).
The theory is centrally Galoistheory, the theory which was developed to prove that there is no general algebraic formula for the roots of a polynomial of degree five.\\\\
