The earliest trace of number theoretic nature is the Babylonian clay tablet Plimpton 322 (ca 1800 BC) which contains a list of Pythagorean triples, that is positive integers $a,b,c$ satisfying $a^2+b^2=c^2$. It is believed, because of the sheer amount of tuples\textemdash that it was not constructed with brute force, but with a method \cite{plimpton322:collection}.
Around 300 BC the ancient Greek mathematician and logician Euclid collected most of mathematics of the time and condensed it to a series of 13 books known as \textit{Euclid's Elements}. The books are complete with postulates, propositions and definitions, with a high regard to rigour. The treatise dealt with geometry, number theory and conics. \cite{euclid:notgood}. \\
In book VII, Euclid presents an algorithm now called Euclid's algorithm. In a lecture given by Johan Jonasson at Chalmers University of Technology he introduces the algorithm as follows:\\
Let $d=\gcd{(a,b)}$ denote the greatest common divisor of the positive integers $a,b$.\\
Then by successively computing the quotients and remainders for the sequence
\begin{align*}
	&a=q_1b+r_1,\quad 0<r_1\leq b-1\\
	&b=q_2r_1+r_2,\quad 0<r_2\leq r_1-1\\
	&r_1=q_3r_2+r_3,\quad 0<r_3\leq r_2-1\\
	&\vdots\\
	&r_{n-2}=q_nr_{n-1}+r_n,\quad 0<r_n\leq r_{n-1}-1\\
	&r_{n-1}=q_{n+1}r_n.
\end{align*}
until the remainder is zero, then the greatest common divisor is given by $r_n=\gcd{(a,b)}$ \cite{talteori1}, see theorem 207 in \cite{hardy} for a proof that the algorithm yields the greatest common divisor.\\
\iftrue
\noindent
The algorithm terminates in less than $2\log_2{b}+1$ steps, the following is a proof sketch:
\begin{align*}
&\begin{cases}
	r_{k-1}=q_{k+1}r_{k}+r_{k+1},\quad k>1\\
	r_k=q_{k+2}r_{k+1}+r_{k+2}
\end{cases}
\implies r_{k-1}=q_{k+1}(q_{k+2}r_{k+1}+r_{k+2})+r_{k+1}\\
&>q_{k+1}q_{k+2}r_{k+1}+r_{k+1}>2r_{k+1}
\end{align*}
Thus, we have the recursive relation $r_k>2r_{k+2}$ for $k=0,1,\ldots,n$, where $b=r_0$. Let $i=\left\lfloor\frac{n}{2}\right\rfloor$, then
\begin{align*}
	&b>2r_2>4r_4>\ldots>2^ir_{2i}>2^i\\
	&\implies \log_2{b}>i\geq \frac{n}{2}
\end{align*}
Thus $2\log_2b$ is an upper bound for $n$.\\
\fi

%Between the 17th and mid 18th century substantial development of the field continued most notably by Fermat, Euler, Lagrange, Legendre and Gauss. 

In another lecture by Jonasson he continues talking about number theory. In particular he defines the Euler Totient function $\phi(n)$ which counts the numbers below $n$ that are relatively prime to $n$, or formally 
$$\phi(n)=\left|\{0 < x < n \mid \gcd{(x,n)=1}\}\right|.$$
For this function the following three properties hold:

\enuma{
	\item If $p$ is prime then $\phi(p)=p-1$
	\item If $\gcd{(a,b)}=1$ then $\phi(ab)=\phi(a)\phi(b)$
	\item If $p,q$ are prime then $\phi(pq)=(p-1)(q-1)$
}
Furthermore, Jonasson explains and proves Euler's Theorem:\\
If $\gcd{(a,n)}=1$ then
$$a^{\phi{(n)}}\equiv 1 \pmod{n}$$
As the final part of the lecture, he explains the RSA (Rivest-Shamir-Adleman) cryptosystem as an application of number theory used for secure communications. For example, Amy wishes to send a message to Bob such that
\enum{
\item Nobody but Bob can read it.
\item Bob will know it was Amy who sent it.
\item Nobody can tamper with the message.
}
This is something RSA can achieve. Jonasson explains that each user must choose two very large distinct primes $p,q$ and a natural number $a$ such that 
$$\gcd{(a,\phi{(pq)})}=\gcd{(a,(p-1)(q-1))}=1$$
and then reveal $pq$ and $a$ (public key). It is unfeasible to find $p$ and $q$ given $pq$ because prime number factorization is rather slow, thus finding $\phi{(pq)}$ also presents difficulty, Jonasson states. \\
Suppose Amy wants to send a message to Bob, then, let $x$ be an integer less than $p,q,a$ and $b$ where $b$ is an integer satisfying $ab\equiv 1 \pmod{\phi{(pq)}}$.\\
Amy encrypts and sends
$$y=x^a \mod pq$$
and then Bob computes
$$x=y^b \mod{pq}$$
and thus the message has been decrypted.
\cite{talteori2}.
\\\\
A theorem called the Prime Number Theorem (PNT) states that the prime counting function $$\pi(x)= \text{\# primes} \leq x$$ satisfies the equality $$\lim_{x\to \infty}\pi(x) \cdot \frac{\ln x}{x}=1.$$
or equivalently $\pi(x) = O\left(\dfrac{x}{\ln x}\right)$.
This proposition was first proved in 1896 using complex analysis. In 1921 G.H Hardy wrote in a letter that he believes a proof of PNT not fundamentally relying on the theory of functions is extraordinary unlikely. However, in 1948 Paul Erdős and Atle Selberg proved PNT without complex analysis \cite{Goldfeld2004}\cite{numbertheory:analytic}. The proof was elementary per definition, but was in fact far more technical than previous proofs. \\

At a talk Christian Johansson, a mathematician and a lecturer at the University of Gothenburg and Chalmers Technical University talks about what mathematicians do and how the science is researched. Johansson studies number theory with algebraic structures, in particular Langlands theory.

%// Erdös number in "Google sökmotor- linjär algebra"?

