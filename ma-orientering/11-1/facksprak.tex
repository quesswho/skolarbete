\chapter{Fackspråk}
\section{Sammanfattning}
\begin{itemize}
	\item ca. 6000-9000 tecken
\end{itemize}
\noindent
Ex. För att presentera Optimeringsteori är det lämpligt att referara till en definitions källa.\\
Jämför med hur en parafras också är ett konstbegrepp.\\
Håll texten objektiv, undvik ord som \textbf{intressant} och \textbf{bra} ty, de är åsikter. \\
"I sin föreläsning under my-dagen (2011-10-24) talar Maja Johanson om hur användbart matematik kan vara." bättre än att använda markörer som ${}^{[1]}$\\
Upprepad påminnelse om att det är någon annan som skriver, inte mig.\\
\section{Hur?}
\begin{itemize}
	\item Påståenden/idéer från föreläsningarna får inte existera i ett vakuum.
	\item Dra paralleller, hitta kopplingar och var tydlig för läsarens skull.
	\item Kom ihåg det övergripandet syftet med din text \textemdash och upprepa det gärna.
\end{itemize}
Bra fraser:
\begin{itemize}
	\item Ett annat exempel ...
	\item En central idé är ...
\end{itemize}
\section{Övergångar}
Övergångar till andra föreläsningar är viktiga.\\
\begin{itemize}
	\item Matematisk optimering kan användas till mer än...
	\item Precis som optimering kan statistik tillämpas inom ...
	\item Currys och åblads presentationer handlade båda om hur de använder optimering i yrkesliv...
\end{itemize}
Aldrig skriv endast förnamn.
Ämnesintroduktion och upprepning av syfte.
Fördjupning/Precisering
Vi ska förklara överdrivet för att examinatorn ska se att skrivaren förstår.
Behöver en inledning och en avslutning.
I inledningen ska syftet framställas. Undvik klyschor av typen "Redan de gamla grekerna insåg att matematiken spelar en avgörande roll ..."\\
Avslutningen, kan vara, "Matematik är \textbf{alltså} en .."
\begin{itemize}
	\item Väj en lagom personlig ton
	\item Disponera texten på ett logiskt sätt
	\item Förklara allt som behöver förklaras
	\item Skriv informativa rubriker
	\item Undvik långa och invecklade meigar
	\item Anänd begripliga ord och förklara nödvändiga facktermer.
\end{itemize}
Ordet vi inom ett matematiskt sammanhang placerar läsaren som en kollega.
I texten, skriv inte för mycket pronomen, bara om det är motiverat.\\
Använd inte rubriker ska inte användas istället för övergångar.\\
Tydliga markörer såsom, "å andrasidan", "o ena sidan", "för det första", "för det andra"