\chapter{Victor Fingal - Volvo}
\enum{
	\item Systemutvecklare med tillämpningar inom kredisk och regel.
	\item Volvofinans bank:\\
	$\cdot$ Grundades 1959.
}
	Victor jobbar inom avdelningen IRK. Del av business intelligence, förvaltar :
	En vanlig dag:
	8:00 Bevaka flöden
	8:45 standup teamet samlas och berättar hur arbetet går.
	9 - Jobba med de projekt som jag är tilldelad.
	Eventuella möten med projektledare eller andra avdelningar.
	Längsta project > 300 timmar.
	\section{Basel och IRK}
	Alla bankaer inom EU följder Baselkomittens standared för hantering av kreditisk och kapitaltöcknin.
	Inom ramverket ges möjligheten för banken tt använda statiska metoder.
	\section{Vad menar vi med Kredirisk}
	Probability of default.
	Loss given default
	Eposure at default
	
	$$ECL =\frac{PD - LGD - EAD}{{(l-r)}^2}$$
	
	$$UL = f(EAD,LGD,PD),VaR=EL+UL$$
	
	\section{Modellerna}
	\enum{
	\item Transparens är väldigt viktigt inom Basel ramverket
	\item Modellerna är utvecklade av oss men godkänns av FI.
	\item Validering av modeller sker löpande. Modellerna "finjusteras".
	\item Banken jämförs mot alla andra banker inom EU.
}
	\section{Tips och råd}
	Kurser:
	\enum{
\item Topologi
\item  Logisk teori
\item Ikelinjär optimering
\item Fourieranalys
\item Finansiell risk	
}
Råd:
\enum{
\item Kom ihåg att läsa kurser ni tycker är kul. Men var lite taktiska.
\item Även det man inte använder är bra.
\item  Om ni vill jobba med data så hjälper det att veta vad eb databas är..
}

